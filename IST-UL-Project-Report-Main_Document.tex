\documentclass[11pt,a4paper,oneside]{report}
\input{./IST-UL-Project-Report-Preamble}
\raggedbottom
\begin{document}
\pagestyle{plain}
% Set roman numbering (i,ii,...) before the start of chapters
\pagenumbering{roman}

% #############################################################################     % This is the FRONT COVER of the IST-UL-Project-Report TEMPLATE. 
% !TEX root = ./main.tex
% #############################################################################
% Version 1.0, October 2018
% BY: Prof. Rui Santos Cruz, rui.s.cruz@tecnico.ulisboa.pt
% #############################################################################
% #############################################################################
% DO NOT CHANGE THE FOLLOWING 4 LINES
\thispagestyle {empty}
\includegraphics[width=5cm]{./pictures/IST_A_RGB_POS.png}
\begin{center}
\vspace{5.0cm}
% #############################################################################
% #############################################################################
%
% INSERT THE TITLE OF THE PROJECT HERE
{\FontLb Housing Option for Permanent Living and Local Accommodation} \\
\vspace{0.2cm}
%
% INSERT THE SUBTITLE OF THE REPORT HERE
{\FontMn Decision Support Models} \\
\vspace{1.0cm}
\vspace*{1.0cm}
\begin{center}
\begin{tabular}{r@{~}l l}
    \multicolumn{3}{c}{\bfseries\textbf{ }} \\
    % INSERT YOUR TEAM NUMBER HERE
    & \textbf{Team nr.}: & NN \\
    % INSERT IDs and NAMES of STUDENTS 
    & 97380  & Francisco Azeredo \\
    & 99021  & Maria Carolina Pessoa \\
    & 106918 & Margarida Fidalgo \\
    & 107089 & Catarina Rodrigues \\
    & 107539 & Débora Roque\\
     % Comment if not necessary
\end{tabular}
\end{center}
\vspace*{2.0cm} \\
{\FontMn Master’s degree in Industrial Engineering and Management} 
{\FontMb IST-TagusPark} \\
\vspace{1.5cm}
{\FontMb 2025/2026} \\
\end{center}
\cleardoublepage


\pagenumbering{arabic}


\chapter{Exercise 1: Influence diagrams and Decision Trees}
\section{Question 1a}
This project analyses a strategic decision faced by H2People regarding the commercialisation of an innovative hydrogen electrolyser. Despite its high efficiency, elevated production costs create uncertainty about market acceptance. The decision is supported using influence diagrams and decision trees to compare the available alternatives under uncertainty.


The decision problem is framed following Hammond, Keeney, and Raiffa's methodology~\cite{Hammond1999}. H2People faces three strategic options: independent production, partnership, or patent sale. The key uncertainty is market demand (strong, moderate, or weak sales), each with associated probabilities. Financial consequences are measured in Net Present Value (NPV). This structure enables formal analysis to identify the strategy maximising expected value while assessing risk attitudes.

The influence diagram (Fig.~2) includes a decision node (\emph{Production Strategy}), a chance node (\emph{Sales}), and a value node (\emph{NPV}). The decision tree (Fig.~3) yields EMVs: independent production (309.85 M€), patent sale (180 M€), and partnership (154.93 M€). Under risk neutrality, independent production is optimal.

\section{Question 1b}
With perfect information on market outcomes, H2People can select the optimal strategy for each scenario, yielding an EMV of 417.25 M€ (Fig.~4).

The Expected Value of Perfect Information (EVPI) is obtained as the difference between the EMV with perfect information and the EMV of the original problem:
\begin{equation}
\text{EVPI} = 417.25 - 309.85 = 107.40 \, \text{M€}
\end{equation}

\section{Question 1c}

Adding a Market Research Report node (favourable, neutral, unfavourable) with posterior probabilities from Bayes' theorem yields EMV = 351.51 M€. The optimal strategy: produce independently (favourable/neutral reports) or sell the patent (unfavourable).

\section{Question 1d}

\begin{equation}
\text{ EVII} = 351.51 - 309.85 = 41.66 \text{ M€}
\end{equation}
The Expected Value of Imperfect Information (EVII):

\begin{equation}
\text{EVII} = 351.51 - 309.85 = 41.66 \text{ M€}
\end{equation}

The study should only be commissioned if costs remain below 41.66 M€.

\section{Question 1e}

With exponential utility function ($R = 10$ M€: $U(x) = 1 - e^{-x/10}$), the certainty equivalent is 181 M€ (vs. EMV: 351.51 M€). The optimal strategy shifts to patent sale regardless of market research, as the utility function penalises the potential 177 M€ loss under weak sales.


\chapter{Exercise 2: Monte Carlo Simulation}


\section*{Question 2 -- Monte Carlo Simulation}

In this question, a Monte Carlo simulation is used to assess the risk associated with producing the electrolyser. Unlike the previous deterministic analysis, uncertainty is explicitly incorporated in the key profitability drivers: sales volume, selling price, unit cost, and fixed costs.

By modelling these variables as independent stochastic inputs, the simulation estimates the probability distribution of first-year profit, providing a more realistic representation of expected performance and downside risk. The results support the subsequent statistical and sensitivity analyses.

\section{Question 2a}

\subsection*{Question 2a) Monte Carlo Simulation Results}

Expected Profit increases with price: 44.2 M€ (1.15 M€/unit) → 122.5 M€ (1.33 M€/unit), with higher variability at higher prices. The 1.33 M€/unit scenario shows positive 5th percentile and superior downside protection, making it the most attractive alternative.

\section{Question 2b}

The tornado diagram obtained from the Monte Carlo simulation provides a sensitivity ranking of the uncertain input variables according to their influence on the first-year Profit ($Y_1$). The analysis is based on regression coefficients, which measure both the strength and the direction of the relationship between each input variable and the model output.

Profit = Sales Volume × (Selling Price − Unit Cost) − Fixed Costs. Sales Volume dominates profitability uncertainty (coefficients 0.74−0.86), while Unit Cost and Fixed Costs have secondary negative effects (−0.44 to −0.33). This indicates that demand variability, not cost control, is the primary profit driver. Higher prices reduce the relative impact of fixed costs.

\section{Question 2c}

After analysing profit considering three fixed selling price scenarios, the company decided to assess the impact on profit by assuming that the selling price follows the best price of competitors. Consequently, the selling price is now described by a Triangular distribution with parameters $E = 1.15$, $F = 1.17$, and $G = 1.33$ (in M\euro{} per unit). Additionally, after conducting market research, H2People concluded that the selling price and sales volume could be correlated with a coefficient of $-0.7$, reflecting the economic reality that higher prices tend to reduce demand.

Table~\ref{tab:profit_2c} presents the statistical measures for the first-year Profit considering the new selling price distribution. Since the market research only indicates the possibility of correlation between selling price and sales volume, statistical measures were obtained for situations both with and without correlation, enabling a better understanding of how this correlation affects profit under the new scenario established in this question.

According to the simulation results presented in Figure~\ref{fig:profit_dist_corr}, the mean profit obtained is 76.42 M\euro{}, with a standard deviation of 48.41 M\euro{}. The 5th percentile is located at 1.2 M\euro{} and the 95th percentile at 160.2 M\euro{}. The probability of obtaining negative profit is 0.050, or approximately 5.0\%, indicating a relatively low risk of incurring losses in the first year of operation.
With triangular price distribution (E=1.15, F=1.17, G=1.33 M€/unit) and −0.7 correlation between price and sales volume, the mean profit is 76.42 M€ (SD: 48.41 M€), 5th−95th percentile: 1.2−160.2 M€, with 5.0% loss probability.

The profit distribution is approximately normal (skewness: 0.2554, kurtosis: 2.8393) with heavier tails. The negative correlation between price and sales creates a stabilising effect, reducing profit variability through natural hedging: when price increases, sales decrease, and vice versa.

Accounting for correlation decreases standard deviation by 18\% (58.76 → 48.41 M€) and increases the 5th percentile from −8.77 to 1.2 M€, reducing loss probability from 7.4\% to 5.0\%. This narrower range (1.2−160.2 M€) increases forecast reliability.

Statistics resemble the 1.17 M€/unit scenario (the triangular distribution mode). Tornado diagrams confirm sales volume as the dominant driver in both correlated and non-correlated cases. Correlation substantially reduces selling price's individual impact through natural hedging.

\section{Question 2d}

For a more comprehensive evaluation of the project's economic viability, the first-year profit calculated previously was used as the basis for projecting cash flows over a four-year time horizon. The cash flows for subsequent years (years 2, 3, and 4) were modelled considering a stochastic annual growth rate, represented by a Normal distribution with mean 2\% and standard deviation 0.5\%. This approach allows capturing the inherent uncertainty in the future evolution of the business, recognising that growth is not deterministic but subject to market variability.

The calculation of the Net Present Value (NPV) considered an initial investment of 100 M\euro{} in year zero and a discount rate of 10\%, reflecting the opportunity cost of capital and the risk associated with the project. The formula applied was:

\begin{equation}
\text{NPV} = \sum_{t=1}^{4} \frac{CF_t}{(1 + r)^t} - \text{Initial Investment}
\label{eq:npv_formula}    ~~\cite[2]
\end{equation}

where $r = 0.10$ is the discount rate and the cash flows for each year were calculated by successively applying the stochastic growth rate to the first-year profit:

\begin{align}
CF_2 &= CF_1 \times (1 + g_2) \label{eq:cf2} \\
CF_3 &= CF_2 \times (1 + g_3) \label{eq:cf3} \\
CF_4 &= CF_3 \times (1 + g_4) \label{eq:cf4}
\end{align}

where $g_t \sim N(0.02, 0.005^2)$ represents the stochastic growth rate for year $t$.

The Monte Carlo simulation enabled obtaining the probabilistic distribution of NPV, providing a comprehensive view of the risk and return associated with the project. The results obtained are presented in Table~\ref{tab:npv_stats}.

Mean NPV (149.02 M\euro{}) suggests value creation, but high variability (SD: 157.76 M\euro{}) and 18.2\% loss probability reflect substantial uncertainty. The 5th percentile (-96 M\euro{}) indicates significant downside risk. The NPV distribution is approximately normal (skewness: 0.2554, kurtosis: 2.8393).

The tornado diagram confirms sales volume as the primary NPV driver (range: -200 to 320 M\euro{}), with unit costs and fixed costs as secondary factors. Growth rate and selling price have smaller impacts. This emphasises that market share strategies should be prioritised over marginal cost reductions for project viability.

\section{Question 2e}

For risk-averse H2People ($R = 10$ M\euro{}), an exponential utility function $U(x) = 1 - e^{-x/R}$ incorporates asymmetric loss sensitivity, strongly penalising negative outcomes.

The Certainty Equivalent ($CE = -R \times \ln(\mathbb{E}[e^{-x/R}])$) is -224.29 M\euro{}, indicating that risk aversion with $R = 10$ M\euro{} makes accepting the project equivalent to a certain loss despite positive mean NPV. High variability (SD: 157.76 M\euro{}) and 18.2\% loss probability cause the utility function to penalise downside scenarios so heavily that expected utility becomes negative. This is consistent with Question 1e, where $R = 10$ M\euro{} shifted the optimal strategy from independent production to patent sale, eliminating downside risk entirely.

With $CE < 0$, the project should not proceed in its current form. Despite positive mean NPV (149.02 M\euro{}) and 81.8\% success probability, the risk significantly exceeds H2People's tolerance. Alternative strategies include risk-sharing partnerships, phased implementation with abandonment options, external financing mechanisms, or patent sale as a risk-free alternative.

\appendix

\chapter{Tables}

\begin{table}[H]
\centering
\caption{Statistical measures of first-year Profit for different selling price scenarios (Question 2a)}
\label{tab:profit_stats}
\begin{tabular}{lccc}
\hline
\textbf{Scenario price (M€/unit)} & \textbf{1.15} & \textbf{1.17} & \textbf{1.33} \\
\hline
Mean  & 44.24 & 52.94 & 122.54 \\
Standard deviation & 49.11 & 50.69 & 64.42 \\
5th percentile  & -30.01 & -23.79 & 24.61 \\
95th percentile  & 131.82 & 143.59 & 237.98 \\
$P(\text{Profit}<0)$ & 0.201 & 0.207 & 0.213 \\
\hline
\end{tabular}
\end{table}

\begin{table}[H]
\centering
\caption{Statistical measures of first-year Profit (Question 2c)}
\label{tab:profit_2c}
\begin{tabular}{lcc}
\hline
\textbf{Measure} & \textbf{With Correlation} & \textbf{Without Correlation} \\
\hline
Mean & 76.42 M\euro{} & 79.13 M\euro{} \\
Standard Deviation & 48.41 M\euro{} & 58.76 M\euro{} \\
5th Percentile & 1.2 M\euro{} & -8.77 M\euro{} \\
95th Percentile & 160.2 M\euro{} & 182.2 M\euro{} \\
P(Profit $<$ 0) & 0.050 (5.0\%) & 0.074 (7.4\%) \\
\hline
\end{tabular}
\end{table}

\begin{table}[H]
\centering
\caption{Statistical measures of Net Present Value (NPV) (Question 2d)}
\label{tab:npv_stats}
\begin{tabular}{lc}
\hline
\textbf{Measure} & \textbf{Value} \\
\hline
Mean NPV & 149.02 M\euro{} \\
Standard Deviation & 157.76 M\euro{} \\
5th Percentile & -96 M\euro{} \\
95th Percentile & 422 M\euro{} \\
P(NPV $<$ 0) & 0.182 (18.2\%) \\
\hline
\end{tabular}
\end{table}

\begin{table}[H]
\centering
\caption{Utility Analysis and Certainty Equivalent (Question 2e)}
\label{tab:utility_analysis}
\begin{tabular}{lc}
\hline
\textbf{Parameter} & \textbf{Value} \\
\hline
Risk Tolerance ($R$) & 10.000 M\euro{} \\
Mean NPV & 149.02 M\euro{} \\
Certainty Equivalent (CE) & -224.29 M\euro{} \\
Decision & Do not proceed \\
\hline
\end{tabular}
\end{table}

\chapter{Figures}

\begin{figure}[H]
    \centering
    \includegraphics[width=0.5\linewidth]{Q1/diagramaa).png}
    \caption{Influence Diagram (Question 1a)}
    \label{fig:influenceDiagrama}
\end{figure}

\begin{figure}[H]
    \centering
    \includegraphics[width=0.75\linewidth]{Q1/DecisionTreea).png}
    \caption{Decision tree (Question 1a)}
    \label{fig:DecisionTreea}
\end{figure}

\begin{figure}[H]
    \centering
    \includegraphics[width=0.75\linewidth]{Q1/diagramab).png}
    \caption{Influence Diagram (Question 1b)}
    \label{fig:influencediagramb}
\end{figure}

\begin{figure}[H]
    \centering
    \includegraphics[width=0.85\linewidth]{Q1/precisiontreeb.png}
    \caption{Decision Tree (Question 1b)}
    \label{fig:decisionTreeb}
\end{figure}

\begin{figure}[H]
\centering
\includegraphics[width=0.6\textwidth]{Profit.png}
\caption{Probability distribution of first-year Profit obtained from the Monte Carlo simulation (Question 2a)}
\label{fig:profit_dist}
\end{figure}

\begin{figure}[H]
\centering
\includegraphics[width=0.6\textwidth]{Cumulative.png}
\caption{Ascending cumulative distribution (Question 2a)}
\label{fig:cumulative_dist}
\end{figure}

\begin{figure}[H]
\centering
\includegraphics[width=0.6\textwidth]{tornado_regression.png}
\caption{Tornado diagram for first-year Profit (Question 2b)}
\label{fig:tornado_regression}
\end{figure}

\begin{figure}[H]
    \centering
    \includegraphics[width=0.7\linewidth]{image.png}
    \caption{Probability distribution of first-year Profit considering correlation between selling price and sales volume (Question 2c)}
    \label{fig:profit_dist_corr}
\end{figure}

\begin{figure}[H]
    \centering
    \includegraphics[width=0.7\linewidth]{WhatsApp Image 2026-01-16 at 17.33.41.jpeg}
    \caption{Probability distribution of first-year Profit without considering correlation between selling price and sales volume (Question 2c)}
    \label{fig:profit_dist_no_corr}
\end{figure}

\begin{figure}[H]
\centering
\includegraphics[width=0.7\textwidth]{WhatsApp Image 2026-01-16 at 17.50.21 (1).jpeg}
\caption{Tornado diagram for first-year Profit with correlation (Question 2c)}
\label{fig:tornado_2c_corr}
\end{figure}

\begin{figure}[H]
\centering
\includegraphics[width=0.7\textwidth]{WhatsApp Image 2026-01-16 at 17.37.28.jpeg}
\caption{Tornado diagram for first-year Profit without correlation (Question 2c)}
\label{fig:tornado_2c_no_corr}
\end{figure}

\begin{figure}[H]
    \centering
    \includegraphics[width=0.7\linewidth]{WhatsApp Image 2026-01-16 at 17.42.04.jpeg}
    \caption{Probability distribution of Net Present Value (NPV) (Question 2d)}
    \label{fig:npv_dist}
\end{figure}

\begin{figure}[H]
    \centering
    \includegraphics[width=0.7\linewidth]{WhatsApp Image 2026-01-16 at 17.42.04 (1).jpeg}
    \caption{Tornado diagram for Net Present Value (NPV) (Question 2d)}
    \label{fig:tornado_npv}
\end{figure}

\begin{thebibliography}{1}
\bibitem{Hammond1999}
J.~S. Hammond, R.~L. Keeney, and H.~Raiffa,
\textit{Smart Choices: A Practical Guide to Making Better Decisions},
Harvard Business School Press, Boston, MA, 1999, Chapter~7.
\end{thebibliography}

\end{document}
% #############################################################################

