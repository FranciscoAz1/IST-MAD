\documentclass[11pt,a4paper,oneside]{report}
\input{./IST-UL-Project-Report-Preamble}
\raggedbottom
\begin{document}
\pagestyle{plain}
% Set roman numbering (i,ii,...) before the start of chapters
\pagenumbering{roman}

% #############################################################################     % This is the FRONT COVER of the IST-UL-Project-Report TEMPLATE. 
% !TEX root = ./main.tex
% #############################################################################
% Version 1.0, October 2018
% BY: Prof. Rui Santos Cruz, rui.s.cruz@tecnico.ulisboa.pt
% #############################################################################
% #############################################################################
% DO NOT CHANGE THE FOLLOWING 4 LINES
\thispagestyle {empty}
\includegraphics[width=5cm]{./pictures/IST_A_RGB_POS.png}
\begin{center}
\vspace{5.0cm}
% #############################################################################
% #############################################################################
%
% INSERT THE TITLE OF THE PROJECT HERE
{\FontLb Housing Option for Permanent Living and Local Accommodation} \\
\vspace{0.2cm}
%
% INSERT THE SUBTITLE OF THE REPORT HERE
{\FontMn Decision Support Models} \\
\vspace{1.0cm}
\vspace*{1.0cm}
\begin{center}
\begin{tabular}{r@{~}l l}
    \multicolumn{3}{c}{\bfseries\textbf{ }} \\
    % INSERT YOUR TEAM NUMBER HERE
    & \textbf{Team nr.}: & NN \\
    % INSERT IDs and NAMES of STUDENTS 
    & 97380  & Francisco Azeredo \\
    & 99021  & Maria Carolina Pessoa \\
    & 106918 & Margarida Fidalgo \\
    & 107089 & Catarina Rodrigues \\
    & 107539 & Débora Roque\\
     % Comment if not necessary
\end{tabular}
\end{center}
\vspace*{2.0cm} \\
{\FontMn Master’s degree in Industrial Engineering and Management} 
{\FontMb IST-TagusPark} \\
\vspace{1.5cm}
{\FontMb 2025/2026} \\
\end{center}
\cleardoublepage


\pagenumbering{arabic}


\chapter{Exercise 1: Influence diagrams and Decision Trees}
\section{Question 1a}
This project analyses a strategic decision faced by H2People regarding the commercialisation of an innovative hydrogen electrolyser. Despite its high efficiency, elevated production costs create uncertainty about market acceptance. The decision is supported using influence diagrams and decision trees to compare the available alternatives under uncertainty.


Before proceeding with the analytical phase, it is necessary to clearly define the decision context by addressing a set of fundamental questions. This structured approach to framing decision problems follows the methodology proposed by Hammond, Keeney, and Raiffa in \textit{Smart Choices} (Chapter~7)~\cite{Hammond1999}.

\paragraph{What are the decisions?}
The decisions available to H2People are to produce the electrolyser independently, to produce it in partnership with another company, or to sell the patent.

\paragraph{What are the key uncertainties?}
The main uncertainty faced by H2People is related to the future market performance of the electrolyser, particularly the level of sales that may be achieved.
\paragraph{What are the possible outcomes of these uncertainties?}
Market performance is represented through three possible outcomes: strong sales, moderate sales, or weak sales.
\paragraph{What are the chances of occurrence of each possible outcome?}
Each sales outcome is associated with a probability of occurrence, reflecting H2People’s expectations based on forecasts and available market information.

\paragraph{What are the consequences of each outcome?}
Each combination of decision and market outcome leads to a specific financial consequence for H2People, expressed in monetary terms such as Net Present Value.

By clearly defining these elements, the decision problem faced by H2People can be formally modelled and analysed, enabling the identification of the strategy that maximises the expected value and, subsequently, the assessment of the influence of risk attitudes on the final decision.

Based on this framework, the first influence diagram was constructed (Fig.~2). The model includes a decision node, a chance node, and a value node. The decision node, \emph{Production Strategy}, represents H2People’s choice between independent production, partnership, or selling the patent. The chance node, \emph{Sales}, captures market uncertainty through three possible outcomes: strong, moderate, and weak demand, each with an associated probability.

The value node, \emph{Net Present Value (NPV)}, represents the financial outcome associated with each combination of production strategy and sales scenario. Expressed in terms of NPV and depicted as a diamond, it provides the criterion for comparing the alternative strategies.

The influence diagram was then converted into a decision tree (Fig.~3), allowing the calculation of the Expected Monetary Value (EMV) for each strategic alternative.


\begin{align}
\text{EMV (Sell the patent)} &= 180 \, \text{M€} \\
\text{EMV (Produce independently)} &= 309.85 \, \text{M€} \\
\text{EMV (Produce through partnership)} &= 154.93 \, \text{M€}
\end{align}

By comparing the EMVs of the three alternatives, producing the electrolyser independently yields the highest Expected Monetary Value. Therefore, under a risk-neutral assumption, the optimal decision is to produce the electrolyser independently.

\section{Question 1b}
To determine the maximum amount H2People should be willing to pay for perfect information on future market performance, a revised decision tree was analysed assuming that sales outcomes are known before the production decision. In the corresponding influence diagram (Fig.~4), an information link is added from the \emph{Sales} chance node to the \emph{Production Strategy} decision node, indicating that the decision is made after the realisation of market conditions. Under perfect information, H2People behaves as a clairvoyant decision-maker, selecting for each sales outcome the strategy that maximises the Net Present Value, thus increasing the expected economic performance. Consequently, for each realised sales outcome, the company selects the production alternative
that maximises the corresponding Net Present Value. The Expected Monetary Value under perfect information is therefore obtained as the probability weighted sum of these best conditional payoffs.

 For this scenario, the decision tree considering perfect information was constructed, resulting in an Expected Monetary Value (EMV) of $417.25\,\text{M€}$.

The Expected Value of Perfect Information (EVPI) is obtained as the difference between the EMV with perfect information and the EMV of the original problem:
\begin{equation}
\text{EVPI} = 417.25 - 309.85 = 107.40 \, \text{M€}
\end{equation}

This value represents the maximum amount the company should be willing to pay for perfect information, since paying more than this amount would eliminate the economic benefit obtained from improved decision-making.

\section{Question 1c}

To incorporate market research into the decision problem, an extended influence diagram was developed by adding a chance node representing the \emph{Market Research Report}, with possible outcomes of favourable, neutral, or unfavourable. An information link connects this node to the \emph{Production Strategy} decision node, indicating that the decision is taken after observing the report outcome. Posterior sales probabilities were obtained using Bayes' theorem.

The resulting decision tree yields an Expected Monetary Value (EMV) of 351.51~M€. The optimal strategy is to produce independently when the report is favourable or neutral, and to sell the patent when it is unfavourable, highlighting the value of market research in reducing exposure to adverse market conditions.

\section{Question 1d}

The maximum amount that H2People should be willing to pay for the market research corresponds to the Expected Value of Imperfect Information (EVII), calculated as the difference between the EMV with market research and the original EMV:

\begin{equation}
\text{ EVII} = 351.51 - 309.85 = 41.66 \text{ M€}
\end{equation}

This value represents the economic benefit of incorporating imperfect information into the decision process. The EVII of 41.66 M€ is substantially lower than the EVPI of 107.40 M€ calculated in Question 1b, reflecting that the market research provides imperfect rather than perfect information. The company should only commission the study if its cost remains below 41.66 M€.

\section{Question 1e}

To assess the impact of risk aversion on the optimal strategy, an exponential utility function was applied with a risk tolerance parameter of $R = 10$ M€, reflecting a significant risk-averse profile:

\begin{equation}
U(x) = 1 - e^{-x/10}
\end{equation}

The decision tree from Question~1c was re-evaluated using certainty equivalents instead of Expected Monetary Values. The resulting certainty equivalent is 181~M€, significantly lower than the EMV of 351.51~M€ obtained under risk neutrality. Moreover, the optimal strategy changes to selling the patent irrespective of the market research outcome.

Under risk aversion with $R = 10$~M€, the exponential utility function strongly penalises adverse outcomes, making the potential loss of 177~M€ under weak sales unacceptable. As a result, the guaranteed return of 181~M€ from selling the patent becomes the preferred strategy, fully eliminating downside risk and confirming the expected behaviour of a risk-averse decision-maker.


\chapter{Exercise 2: Monte Carlo Simulation}


\section*{Question 2 -- Monte Carlo Simulation}

In this question, a Monte Carlo simulation is used to assess the risk associated with producing the electrolyser. Unlike the previous deterministic analysis, uncertainty is explicitly incorporated in the key profitability drivers: sales volume, selling price, unit cost, and fixed costs.

By modelling these variables as independent stochastic inputs, the simulation estimates the probability distribution of first-year profit, providing a more realistic representation of expected performance and downside risk. The results support the subsequent statistical and sensitivity analyses.

\section{Question 2a}

\subsection*{Question 2a) Monte Carlo Simulation Results}

The Monte Carlo simulation was used to obtain the probability distribution of the first-year Profit for the three selling price scenarios considered. Table~\ref{tab:profit_stats} presents the main statistical measures of Profit for each scenario, including the mean, standard deviation, selected percentiles, and the probability of incurring losses.

\begin{table}[h!]
\centering
\caption{Statistical measures of first-year Profit for different selling price scenarios}
\label{tab:profit_stats}
\begin{tabular}{lccc}
\hline
\textbf{Scenario price (M€/unit)} & \textbf{1.15} & \textbf{1.17} & \textbf{1.33} \\
\hline
Mean  & 44.24 & 52.94 & 122.54 \\
Standard deviation & 49.11 & 50.69 & 64.42 \\
5th percentile  & -30.01 & -23.79 & 24.61 \\
95th percentile  & 131.82 & 143.59 & 237.98 \\
$P(\text{Profit}<0)$ & 0.201 & 0.207 & 0.213 \\
\hline
\end{tabular}
\end{table}

The results show that expected Profit increases significantly with the selling price, rising from approximately 44.2~M€ at 1.15~M€/unit to 52.9~M€ at 1.17~M€/unit, and reaching 122.5~M€ in the highest price scenario. This increase is accompanied by higher dispersion, as reflected by the growing standard deviation, highlighting the trade-off between higher expected returns and increased uncertainty.

Downside risk analysis indicates negative 5\% quantiles for the two lowest price scenarios, while the highest price scenario exhibits a positive 5th percentile, suggesting improved protection against extreme losses. Graphical analysis confirms a rightward shift of the Profit distributions and cumulative curves as the selling price increases, with higher prices associated with greater variability but more favourable outcomes overall.

Based on both statistical indicators and graphical evidence, the selling price of 1.33~M€/unit emerges as the most attractive alternative among the scenarios considered.

\section{Question 2b}

The tornado diagram obtained from the Monte Carlo simulation provides a sensitivity ranking of the uncertain input variables according to their influence on the first-year Profit ($Y_1$). The analysis is based on regression coefficients, which measure both the strength and the direction of the relationship between each input variable and the model output.

In this study, Profit is defined as:
\begin{equation}
\text{Profit} = \text{Sales Volume} \times (\text{Selling Price} - \text{Unit Cost}) - \text{Fixed Costs}
\end{equation}

The results indicate that \textit{Sales Volume} is the dominant driver of Profit variability across all analysed selling price scenarios. This variable exhibits strong positive regression coefficients, increasing from approximately 0.74 to 0.86 as the selling price rises. This behaviour demonstrates that variations in sales volume have a substantial and amplified effect on Profit, particularly at higher selling prices.

Both \textit{Unit Cost} and \textit{Fixed Costs} present negative regression coefficients, 
confirming that increases in these variables lead to a reduction in Profit. Their influence is 
significantly smaller than that of Sales Volume and remains secondary across all price 
scenarios. In addition, the magnitude of the coefficients associated with Fixed Costs decreases 
as the selling price rises (from approximately $-0.44$ to $-0.33$), indicating that higher 
selling prices partially mitigate the adverse impact of fixed cost variability on Profit. 
For Unit Cost, although the coefficients remain negative in all scenarios, their magnitude is 
also reduced in the highest price scenario.

Overall, the tornado diagram highlights that Profit uncertainty is primarily driven by 
market-related factors rather than cost-related ones. While cost control remains important, 
the variability of Profit is more strongly affected by uncertainties in demand. Consequently, 
strategies aimed at improving sales performance are likely to have a greater impact on 
profitability than marginal reductions in unit or fixed costs.

\section{Question 2c}

After analysing profit considering three fixed selling price scenarios, the company decided to assess the impact on profit by assuming that the selling price follows the best price of competitors. Consequently, the selling price is now described by a Triangular distribution with parameters $E = 1.15$, $F = 1.17$, and $G = 1.33$ (in M\euro{} per unit). Additionally, after conducting market research, H2People concluded that the selling price and sales volume could be correlated with a coefficient of $-0.7$, reflecting the economic reality that higher prices tend to reduce demand.

Table~\ref{tab:profit_2c} presents the statistical measures for the first-year Profit considering the new selling price distribution. Since the market research only indicates the possibility of correlation between selling price and sales volume, statistical measures were obtained for situations both with and without correlation, enabling a better understanding of how this correlation affects profit under the new scenario established in this question.

\begin{table}[h]
\centering
\caption{Statistical measures of first-year Profit}
\label{tab:profit_2c}
\begin{tabular}{lcc}
\hline
\textbf{Measure} & \textbf{With Correlation} & \textbf{Without Correlation} \\
\hline
Mean & 76.42 M\euro{} & 79.13 M\euro{} \\
Standard Deviation & 48.41 M\euro{} & 58.76 M\euro{} \\
5th Percentile & 1.2 M\euro{} & -8.77 M\euro{} \\
95th Percentile & 160.2 M\euro{} & 182.2 M\euro{} \\
P(Profit $<$ 0) & 0.050 (5.0\%) & 0.074 (7.4\%) \\
\hline
\end{tabular}
\end{table}

According to the simulation results presented in Figure~\ref{fig:profit_dist_corr}, the mean profit obtained is 76.42 M\euro{}, with a standard deviation of 48.41 M\euro{}. The 5th percentile is located at 1.2 M\euro{} and the 95th percentile at 160.2 M\euro{}. The probability of obtaining negative profit is 0.050, or approximately 5.0\%, indicating a relatively low risk of incurring losses in the first year of operation.

Analysing the probability distribution of profit presented in Figure~\ref{fig:profit_dist_corr}, an approximately normal distribution is observed, with slight positive asymmetry (skewness of 0.2554), indicating a higher probability of extreme positive outcomes. The kurtosis of 2.8393 suggests that the distribution has slightly heavier tails than a standard normal distribution, reflecting the presence of some variability at the extremes.

The negative correlation between selling price and sales volume introduces a stabilising effect on profit. When price increases, sales volume tends to decrease, and vice versa, which reduces the overall variability of profit compared to a scenario where these variables would be independent. This behaviour is consistent with the economic theory of price elasticity of demand and provides natural protection against extreme scenarios. Mathematically, this can be expressed as:

\begin{equation}
\text{Profit} = \text{Sales Volume} \times (\text{Selling Price} - \text{Unit Cost}) - \text{Fixed Costs}
\label{eq:profit_formula}
\end{equation}

where the negative correlation between Sales Volume and Selling Price moderates the combined effect on profit.

Examining the values presented in Table~\ref{tab:profit_2c}, it is possible to verify that, although similar, the statistics obtained when considering and when not considering the correlation between sales and selling price show some relevant differences. Firstly, when correlation is accounted for, the standard deviation is reduced by approximately 10.35 M\euro{} (from 58.76 M\euro{} to 48.41 M\euro{}), representing an 18\% reduction in profit dispersion. This indicates a reduction in the dispersion of the values obtained for Profit in each iteration, allowing for greater confidence in the results obtained and the analysis carried out. Next, we can see that the 5th percentile has increased from -8.77 M\euro{} to 1.2 M\euro{}, effectively eliminating most extreme negative scenarios, and the 95th percentile has decreased from 182.2 M\euro{} to 160.2 M\euro{}, once again reflecting the fact that the values obtained for Profit in the different iterations are not so dispersed, staying between 1.2 M\euro{} and 160.2 M\euro{}, with a 90\% probability, when correlation is considered.

Finally, considering the correlation, and as would be expected given the increase in the 5th percentile, the probability of profit being negative decreased by around 2.4 percentage points (from 7.4\% to 5.0\%). With this, we can conclude that although the average profit when correlation is considered is slightly lower, which would be expected since the correlation between selling price and sales is negative, the forecasts made for profit show less variation and are more reliable. In addition, there is less risk of making a loss. Figures~\ref{fig:profit_dist_corr} and~\ref{fig:profit_dist_no_corr} allow us to understand the analysis in a more visual way, and we can observe the greater amplitude between the maximum and minimum values when correlation is not considered.

Comparing the results obtained for the new definition of selling price with the results obtained in Question 2a, where three selling price scenarios are considered, it can be seen that the statistics obtained in this question are similar to the statistics obtained for the second scenario which considers a unit selling price of 1.17 M\euro{}, which would be expected since the triangular distribution used to describe the sales price considers the price of 1.17 M\euro{} as the mode.

Finally, as can be observed in Figures~\ref{fig:tornado_2c_corr} and~\ref{fig:tornado_2c_no_corr}, given that the selling price is no longer fixed for a given scenario and now follows a triangular distribution, variations in selling price also have an impact on Profit. However, sales volume, unit cost, and fixed costs remain the factors with the greatest impact on Profit and maintain the same ranking of importance, with sales volume standing out as the dominant factor. The tornado diagrams illustrate that sales volume exhibits the widest range of influence, followed by unit costs and fixed costs. The inclusion of selling price as a variable adds an additional layer of uncertainty, though its impact remains secondary compared to the volume-related factors.

When comparing the tornado diagrams with and without correlation (Figures~\ref{fig:tornado_2c_corr} and~\ref{fig:tornado_2c_no_corr}), several key differences become evident. In the scenario without correlation, the selling price shows a markedly larger impact on profit variability, with variations spanning a wider range, whereas in the correlated scenario, the selling price impact is substantially diminished. This occurs because the correlation between selling price and sales volume creates a natural hedging effect: when the selling price increases (which would normally increase profit), sales volume decreases (which reduces profit), and vice versa. As a result, the individual impact of selling price on profit variability is partially offset by the inverse movement in sales volume. The tornado diagrams also confirm that sales volume remains the dominant factor in both scenarios, followed by unit costs and fixed costs, though the relative magnitudes differ between the correlated and non-correlated cases. This reinforces the stabilising role of the negative correlation in moderating profit uncertainty and demonstrates how market dynamics can serve as a natural risk mitigation mechanism.

\section{Question 2d}

For a more comprehensive evaluation of the project's economic viability, the first-year profit calculated previously was used as the basis for projecting cash flows over a four-year time horizon. The cash flows for subsequent years (years 2, 3, and 4) were modelled considering a stochastic annual growth rate, represented by a Normal distribution with mean 2\% and standard deviation 0.5\%. This approach allows capturing the inherent uncertainty in the future evolution of the business, recognising that growth is not deterministic but subject to market variability.

The calculation of the Net Present Value (NPV) considered an initial investment of 100 M\euro{} in year zero and a discount rate of 10\%, reflecting the opportunity cost of capital and the risk associated with the project. The formula applied was:

\begin{equation}
\text{NPV} = \sum_{t=1}^{4} \frac{CF_t}{(1 + r)^t} - \text{Initial Investment}
\label{eq:npv_formula}    ~~\cite[2]
\end{equation}

where $r = 0.10$ is the discount rate and the cash flows for each year were calculated by successively applying the stochastic growth rate to the first-year profit:

\begin{align}
CF_2 &= CF_1 \times (1 + g_2) \label{eq:cf2} \\
CF_3 &= CF_2 \times (1 + g_3) \label{eq:cf3} \\
CF_4 &= CF_3 \times (1 + g_4) \label{eq:cf4}
\end{align}

where $g_t \sim N(0.02, 0.005^2)$ represents the stochastic growth rate for year $t$.

The Monte Carlo simulation enabled obtaining the probabilistic distribution of NPV, providing a comprehensive view of the risk and return associated with the project. The results obtained are presented in Table~\ref{tab:npv_stats}.

\begin{table}[h]
\centering
\caption{Statistical measures of Net Present Value (NPV)}
\label{tab:npv_stats}
\begin{tabular}{lc}
\hline
\textbf{Measure} & \textbf{Value} \\
\hline
Mean NPV & 149.02 M\euro{} \\
Standard Deviation & 157.76 M\euro{} \\
5th Percentile & -96 M\euro{} \\
95th Percentile & 422 M\euro{} \\
P(NPV $<$ 0) & 0.182 (18.2\%) \\
\hline
\end{tabular}
\end{table}

The positive mean NPV of 149.02 M\euro{} indicates that, on average, the project is expected to generate substantial financial value, as the expected cash inflows significantly exceed the initial investment when discounted over the project's lifespan. This result is encouraging from a financial perspective and suggests that the project has the potential to create value for the company.

However, the risk analysis reveals important aspects that must be considered. The high standard deviation (157.76 M\euro{}) indicates considerable dispersion of possible outcomes, reflecting substantial uncertainty associated with the project. The probability of 18.2\% of obtaining a negative NPV is not negligible and represents a significant risk that the project may destroy value rather than create it. The 5th percentile of -96 M\euro{} indicates that, in the most unfavourable scenarios (occurring in 5\% of simulations), the project can result in significant losses.

Analysing the probability distribution of NPV presented in Figure~\ref{fig:npv_dist}, an approximately normal distribution is observed with slight positive asymmetry (skewness of 0.2554), suggesting a higher probability of extreme positive outcomes. The kurtosis of 2.8393 indicates that the distribution has slightly heavier tails than a standard normal distribution, reflecting some variability at the distribution extremes.

As can be observed in the tornado diagram presented in Figure~\ref{fig:tornado_npv}, variations in the yearly growth rate also have an impact on NPV, although the factors with the greatest impact on NPV remain sales volume, unit cost, and fixed costs, these being the main factors to consider when making this decision. Sales volume maintains its position as the most influential factor, with the widest range spanning from approximately -200 M\euro{} to 320 M\euro{}, underlining the critical importance of market demand for the financial success of the project. Unit costs and fixed costs follow as secondary drivers, while the selling price and yearly growth rate exhibit comparatively smaller but still meaningful impacts on the final NPV.

The dominance of sales volume in driving NPV variability emphasises that strategies aimed at securing and expanding market share should be prioritised over marginal cost reductions. While cost control remains important, the project's financial viability is fundamentally tied to the company's ability to achieve and sustain adequate sales levels throughout the project lifetime.

In conclusion, although the project presents a positive and attractive mean NPV of 149.02 M\euro{}, the high variability of results and the non-negligible probability of losses (18.2\%) require careful analysis of the company's risk tolerance before making the final investment decision. The company should evaluate its risk appetite and how it aligns with the potential risks and returns of this project, as well as the potential impact of this project on the company's long-term strategy and goals.

\section{Question 2e}

The final analysis of the project incorporates Utility Theory to evaluate the investment decision from the perspective of a company with significant risk aversion. Unlike previous analyses based on Expected Monetary Value (EMV) or mean NPV, which assume risk neutrality, this approach recognises that decision-makers may asymmetrically value potential gains and losses, being particularly sensitive to negative outcomes.

H2People exhibits a risk tolerance of $R = 10$ M\euro{}, reflecting an extremely conservative profile regarding financial uncertainty. To incorporate this preference into the analysis, an exponential utility function was applied with the form:

\begin{equation}
U(x) = 1 - e^{-x/R}
\label{eq:utility_function}
\end{equation}

where $x$ represents the financial outcome (in this case, the project's NPV) and $R = 10$ M\euro{} is the risk tolerance parameter. This function is characterised by strongly penalising negative outcomes, mathematically capturing the company's risk aversion. The smaller the value of $R$, the greater the risk aversion, and in H2People's case, an $R$ of only 10 M\euro{} compared to a project with an initial investment of 100 M\euro{} and potential results ranging between -96 M\euro{} and 422 M\euro{} indicates particularly high risk aversion.

The Certainty Equivalent (CE) was calculated using the formula:

\begin{equation}
CE = -R \times \ln\left(\mathbb{E}\left[e^{-x/R}\right]\right)
\label{eq:certainty_equivalent}
\end{equation}

This value represents the monetary amount that the company would be willing to accept with absolute certainty rather than face the distribution of uncertain project outcomes. According to the simulation performed, the CE obtained is -224.29 M\euro{}.

\begin{table}[h]
\centering
\caption{Utility Analysis and Certainty Equivalent}
\label{tab:utility_analysis}
\begin{tabular}{lc}
\hline
\textbf{Parameter} & \textbf{Value} \\
\hline
Risk Tolerance ($R$) & 10.000 M\euro{} \\
Mean NPV & 149.02 M\euro{} \\
Certainty Equivalent (CE) & -224.29 M\euro{} \\
Decision & Do not proceed \\
\hline
\end{tabular}
\end{table}

This negative value of significant magnitude indicates that, from the company's utility perspective, accepting the project is equivalent to accepting a certain loss of 224.29 M\euro{}, despite the mean NPV being positive (149.02 M\euro{}). This apparent contradiction results from the combination of two critical factors: (1) the company's high risk aversion, represented by a tolerance of only 10 M\euro{}, and (2) the significant dispersion of possible project outcomes, with a standard deviation of 157.76 M\euro{} and a probability of 18.2\% of negative results.

The exponential utility function with $R = 10$ M\euro{} penalises unfavourable scenarios so heavily that the expected value of utility becomes negative, even considering favourable scenarios. Essentially, for a company with this risk aversion profile, the possibility of losing 96 M\euro{} or more (which occurs in 5\% of cases) is so unacceptable that it completely outweighs the potential benefit of positive scenarios.

To better understand this phenomenon, consider that the exponential utility function exhibits increasing absolute risk aversion. For small values of $R$ relative to the scale of potential outcomes, even moderate losses receive disproportionately large negative utility values. When these utilities are averaged across all simulation outcomes, the heavily weighted negative scenarios dominate the calculation, resulting in a negative expected utility and, consequently, a negative certainty equivalent.

This result is consistent with the analysis performed in Question 1e, where a similar risk tolerance of 10 M\euro{} was applied to the original decision problem. In that context, the optimal strategy shifted from producing the electrolyser independently (under risk neutrality) to selling the patent regardless of market research outcomes. The guaranteed return from selling the patent, though lower in expected value terms, became preferable because it eliminated downside risk entirely.

Given that $CE < 0$, as indicated in Table~\ref{tab:utility_analysis}, the clear and unequivocal recommendation is that the project should not proceed in its current form. Although the project presents a positive and significant mean NPV (149.02 M\euro{}) and a reasonable probability of success (81.8\%), the level of risk involved clearly exceeds H2People's risk tolerance. For a company with this risk aversion profile, the uncertainty associated with the project makes it financially unacceptable from the perspective of maximising expected utility.

Alternative strategies that H2People might consider include seeking risk-sharing partnerships to reduce exposure to negative outcomes, implementing the project in phases with decision points to abandon if early results are unfavourable, securing external financing or insurance mechanisms to mitigate downside risk, or reconsidering the sale of the patent as a risk-free alternative, consistent with the recommendation from the decision tree analysis under risk aversion.

\appendix
\chapter{Figures}

\begin{figure}[H]
    \centering
    \includegraphics[width=0.5\linewidth]{Q1/diagramaa).png}
    \caption{Influence Diagram (Question 1a)}
    \label{fig:influenceDiagrama}
\end{figure}

\begin{figure}[H]
    \centering
    \includegraphics[width=0.75\linewidth]{Q1/DecisionTreea).png}
    \caption{Decision tree (Question 1a)}
    \label{fig:DecisionTreea}
\end{figure}

\begin{figure}[H]
    \centering
    \includegraphics[width=0.75\linewidth]{Q1/diagramab).png}
    \caption{Influence Diagram (Question 1b)}
    \label{fig:influencediagramb}
\end{figure}

\begin{figure}[H]
    \centering
    \includegraphics[width=0.85\linewidth]{Q1/precisiontreeb.png}
    \caption{Decision Tree (Question 1b)}
    \label{fig:decisionTreeb}
\end{figure}

\begin{figure}[H]
\centering
\includegraphics[width=0.6\textwidth]{Profit.png}
\caption{Probability distribution of first-year Profit obtained from the Monte Carlo simulation (Question 2a)}
\label{fig:profit_dist}
\end{figure}

\begin{figure}[H]
\centering
\includegraphics[width=0.6\textwidth]{Cumulative.png}
\caption{Ascending cumulative distribution (Question 2a)}
\label{fig:cumulative_dist}
\end{figure}

\begin{figure}[H]
\centering
\includegraphics[width=0.6\textwidth]{tornado_regression.png}
\caption{Tornado diagram for first-year Profit (Question 2b)}
\label{fig:tornado_regression}
\end{figure}

\begin{figure}[H]
    \centering
    \includegraphics[width=0.7\linewidth]{image.png}
    \caption{Probability distribution of first-year Profit considering correlation between selling price and sales volume (Question 2c)}
    \label{fig:profit_dist_corr}
\end{figure}

\begin{figure}[H]
    \centering
    \includegraphics[width=0.7\linewidth]{WhatsApp Image 2026-01-16 at 17.33.41.jpeg}
    \caption{Probability distribution of first-year Profit without considering correlation between selling price and sales volume (Question 2c)}
    \label{fig:profit_dist_no_corr}
\end{figure}

\begin{figure}[H]
\centering
\includegraphics[width=0.7\textwidth]{WhatsApp Image 2026-01-16 at 17.50.21 (1).jpeg}
\caption{Tornado diagram for first-year Profit with correlation (Question 2c)}
\label{fig:tornado_2c_corr}
\end{figure}

\begin{figure}[H]
\centering
\includegraphics[width=0.7\textwidth]{WhatsApp Image 2026-01-16 at 17.37.28.jpeg}
\caption{Tornado diagram for first-year Profit without correlation (Question 2c)}
\label{fig:tornado_2c_no_corr}
\end{figure}

\begin{figure}[H]
    \centering
    \includegraphics[width=0.7\linewidth]{WhatsApp Image 2026-01-16 at 17.42.04.jpeg}
    \caption{Probability distribution of Net Present Value (NPV) (Question 2d)}
    \label{fig:npv_dist}
\end{figure}

\begin{figure}[H]
    \centering
    \includegraphics[width=0.7\linewidth]{WhatsApp Image 2026-01-16 at 17.42.04 (1).jpeg}
    \caption{Tornado diagram for Net Present Value (NPV) (Question 2d)}
    \label{fig:tornado_npv}
\end{figure}

\begin{thebibliography}{1}
\bibitem{Hammond1999}
J.~S. Hammond, R.~L. Keeney, and H.~Raiffa,
\textit{Smart Choices: A Practical Guide to Making Better Decisions},
Harvard Business School Press, Boston, MA, 1999, Chapter~7.
\end{thebibliography}

\end{document}
% #############################################################################

